%!TEX TS-program = xelatex
\documentclass[12pt, a4paper, oneside]{article}

% Можно вставить разную преамбулу
% пакеты для математики
\usepackage{amsmath,amsfonts,amssymb,amsthm,mathtools}  
\mathtoolsset{showonlyrefs=true}  % Показывать номера только у тех формул, на которые есть \eqref{} в тексте.

\usepackage[british,russian]{babel} % выбор языка для документа
\usepackage[utf8]{inputenc}          % utf8 кодировка

% Основные шрифты 
\usepackage{fontspec}         
\setmainfont{Linux Libertine O}  % задаёт основной шрифт документа

% Математические шрифты 
\usepackage{unicode-math}     
\setmathfont[math-style=upright]{Neo Euler} 

\setmathfont[range={\mathbb, \mathop}]{Asana-Math.otf}

%%%%%%%%%% Работа с картинками и таблицами %%%%%%%%%%
\usepackage{graphicx} % Для вставки рисунков                
\usepackage{graphics}
\graphicspath{{images/}{pictures/}}   % папки с картинками

\usepackage[figurename=Картинка]{caption}

\usepackage{wrapfig}    % обтекание рисунков и таблиц текстом

\usepackage{booktabs}   % таблицы как в годных книгах
\usepackage{tabularx}   % новые типы колонок
\usepackage{tabulary}   % и ещё новые типы колонок
\usepackage{float}      % возможность позиционировать объекты в нужном месте
\renewcommand{\arraystretch}{1.2}  % больше расстояние между строками


%%%%%%%%%% Графики и рисование %%%%%%%%%%
\usepackage{tikz, pgfplots}  % языки для графики
%\pgfplotsset{compat=1.16}

\usepackage{todonotes} % для вставки в документ заметок о том, что осталось сделать
% \todo{Здесь надо коэффициенты исправить}
% \missingfigure{Здесь будет Последний день Помпеи}
% \listoftodos --- печатает все поставленные \todo'шки


%%%%%%%%%% Внешний вид страницы %%%%%%%%%%

\usepackage[paper=a4paper, top=20mm, bottom=15mm,left=20mm,right=15mm]{geometry}
\usepackage{indentfirst}    % установка отступа в первом абзаце главы

\usepackage{setspace}
\setstretch{1.15}  % межстрочный интервал
\setlength{\parskip}{4mm}   % Расстояние между абзацами
% Разные длины в LaTeX: https://en.wikibooks.org/wiki/LaTeX/Lengths

% свешиваем пунктуацию
% теперь знаки пунктуации могут вылезать за правую границу текста, при этом текст выглядит ровнее
\usepackage{microtype}

% \flushbottom                            % Эта команда заставляет LaTeX чуть растягивать строки, чтобы получить идеально прямоугольную страницу
\righthyphenmin=2                       % Разрешение переноса двух и более символов
\widowpenalty=300                     % Небольшое наказание за вдовствующую строку (одна строка абзаца на этой странице, остальное --- на следующей)
\clubpenalty=3000                     % Приличное наказание за сиротствующую строку (омерзительно висящая одинокая строка в начале страницы)
\tolerance=10000     % Ещё какое-то наказание.

% мои цвета https://www.artlebedev.ru/colors/
\definecolor{titleblue}{rgb}{0.2,0.4,0.6} 
\definecolor{blue}{rgb}{0.2,0.4,0.6} 
%\definecolor{red}{rgb}{1,0,0.2} 
\definecolor{green}{rgb}{0, 0.6, 0}
\definecolor{purp}{rgb}{0.4,0,0.8} 

\definecolor{red}{RGB}{213,94,0}
\definecolor{yellow}{RGB}{240,228,66}


% цвета из geogebra 
\definecolor{litebrown}{rgb}{0.6,0.2,0}
\definecolor{darkbrown}{rgb}{0.75,0.75,0.75}

% Гиперссылки
\usepackage{xcolor}   % разные цвета

\usepackage{hyperref}
\hypersetup{
	unicode=true,           % позволяет использовать юникодные символы
	colorlinks=true,       	% true - цветные ссылки
	urlcolor=blue,          % цвет ссылки на url
	linkcolor=black,          % внутренние ссылки
	citecolor=green,        % на библиографию
	breaklinks              % если ссылка не умещается в одну строку, разбивать её на две части?
}

% меняю оформление секций 
\usepackage{titlesec}
\usepackage{sectsty}

% меняю цвет на синий
\sectionfont{\color{titleblue}}
\subsectionfont{\color{titleblue}}
\renewcommand{\thesection}{\arabic{section}.}


% выбрасываю нумерацию страниц и колонтитулы 
%\pagestyle{empty}

% синие круглые бульпоинты в списках itemize 
\usepackage{enumitem}

\definecolor{itemizeblue}{rgb}{0, 0.45, 0.70}

\newcommand*{\MyPoint}{\tikz \draw [baseline, fill=itemizeblue, draw=blue] circle (2.5pt);}
\renewcommand{\labelitemi}{\MyPoint}

\AddEnumerateCounter{\asbuk}{\@asbuk}{\cyrm}
\renewcommand{\theenumi}{\asbuk{enumi}}

% расстояние в списках
\setlist[itemize]{parsep=0.4em,itemsep=0em,topsep=0ex}
\setlist[enumerate]{parsep=0.4em,itemsep=0em,topsep=0ex}

% эпиграфы
\usepackage{epigraph}
\setlength\epigraphwidth{.6\textwidth}
\setlength\epigraphrule{0pt}

%%%%%%%%%% Свои команды %%%%%%%%%%

% Математические операторы первой необходимости:
\DeclareMathOperator{\sgn}{sign}
\DeclareMathOperator*{\argmin}{arg\,min}
\DeclareMathOperator*{\argmax}{arg\,max}
\DeclareMathOperator{\Cov}{Cov}
\DeclareMathOperator{\Var}{Var}
\DeclareMathOperator{\Corr}{Corr}

\DeclareMathOperator{\Pois}{Pois}
\DeclareMathOperator{\Geom}{Geom}
\DeclareMathOperator{\Exp}{Exp}

%\DeclareMathOperator{\E}{\mathbb{E}}
\DeclareMathOperator{\Med}{Med}
\DeclareMathOperator{\Mod}{Mod}
\DeclareMathOperator*{\plim}{plim}

% команды пореже
\newcommand{\const}{\mathrm{const}}  % const прямым начертанием
\newcommand{\iid}{\sim i\,i\,d\,\,}  % ну вы поняли...
\newcommand{\fr}[2]{\ensuremath{^{#1}/_{#2}}}   % особая дробь
\newcommand{\ind}[1]{\mathbbm{1}_{\{#1\}}} % Индикатор события
\newcommand{\dx}[1]{\,\mathrm{d}#1} % для интеграла: маленький отступ и прямая d

% одеваем шапки на частые штуки
\def \hb{\hat{\beta}}
\def \hs{\hat{s}}
\def \hy{\hat{y}}
\def \hY{\hat{Y}}
\def \he{\hat{\varepsilon}}
\def \hVar{\widehat{\Var}}
\def \hCorr{\widehat{\Corr}}
\def \hCov{\widehat{\Cov}}

% Греческие буквы
\def \a{\alpha}
\def \b{\beta}
\def \t{\tau}
\def \dt{\delta}
\def \e{\varepsilon}
\def \ga{\gamma}
\def \kp{\varkappa}
\def \la{\lambda}
\def \sg{\sigma}
\def \tt{\theta}
\def \Dt{\Delta}
\def \La{\Lambda}
\def \Sg{\Sigma}
\def \Tt{\Theta}
\def \Om{\Omega}
\def \om{\omega}

% Готика
\def \mA{\mathcal{A}}
\def \mB{\mathcal{B}}
\def \mC{\mathcal{C}}
\def \mE{\mathcal{E}}
\def \mF{\mathcal{F}}
\def \mH{\mathcal{H}}
\def \mL{\mathcal{L}}
\def \mN{\mathcal{N}}
\def \mU{\mathcal{U}}
\def \mV{\mathcal{V}}
\def \mW{\mathcal{W}}

% Жирные буквы
\def \mbb{\mathbb}
\def \RR{\mbb R}
\def \NN{\mbb N}
\def \ZZ{\mbb Z}
\def \PP{\mbb{P}}
\def \E{\mbb{E}}
\def \QQ{\mbb Q}

%%%%%%%%%% Теоремы %%%%%%%%%%
\theoremstyle{plain} % Это стиль по умолчанию.  Есть другие стили.
\newtheorem{theorem}{Теорема}[section]
\newtheorem{proposition}{Утверждение}[section]
\newtheorem{result}{Следствие}[theorem]
% счётчик подчиняется теоремному, нумерация идёт по главам согласованно между собой

% убирает курсив и что-то еще наверное делает ;)
\theoremstyle{definition}         
\newtheorem*{definition}{Определение}  % нумерация не идёт вообще


%%%%%%%%%% Задачки и решения %%%%%%%%%%
\usepackage{etoolbox}    % логические операторы для своих макросов
\usepackage{environ}
\newtoggle{lecture}

\newcounter{probNum}[section]  % счётчик для упражнений 
\NewEnviron{problem}[1]{%
    \refstepcounter{probNum}% увеличели номер на 1 
    {\noindent \textbf{\large \color{titleblue} Упражнение~\theprobNum~#1}  \\ \\ \BODY}
    {}%
  }

% Окружение, чтобы можно было убирать решения из pdf
\NewEnviron{sol}{%
  \iftoggle{lecture}
    {\noindent \textbf{\large Решение:} \\ \\ \BODY}
    {}%
  }
 
% выделение по тексту важных вещей
\newcommand{\indef}[1]{\textbf{ \color{green} #1}} 

\usepackage[normalem]{ulem}  % для зачекивания текста

% Если переключить в false, все solution исчезнут из pdf
\toggletrue{lecture}
%\togglefalse{lecture}



% для нормального распределения
\newcommand{\expp}[1]{ \exp \left( #1 \right)} 
% для прорисовки нормального распределения
\newcommand\gauss[2]{1/(#2*sqrt(2*pi))*exp(-((x-#1)^2)/(2*#2^2))} 


\title{\begin{center} \includegraphics[width=0.99\textwidth]{logo.png} \end{center}  Каждой Маше по три медведя! \footnote{Эта pdf-ка, по факту, представляет из себя кусочек недописанной виньетки по Байесовским методам: \newline  \url{https://github.com/FUlyankin/book_about_bayes}}}
% \author{Ульянкин Филя, Романенко Саша}

\begin{document}
	
	\maketitle
	
\epigraph{Идет медведь по лесу, видит, машина горит. Сел в нее и сгорел.}{Анекдот категории F}
	
\begin{problem}{(Маша и медведи)}
Маша прячется от Медведей в точке $m$ на числовой прямой. Есть несколько Медведей, каждый из которых обнюхивает всю числовую прямую в поисках Маши. Медведю номер $i$ кажется, что Машей сильней всего пахнет в точке $y_i$. Естественно, Медведи могут ошибаться, например, у них может быть заложен нос, поэтому \indef{модель Медведя выглядит как:}

\[ y_i \mid m \sim \mN(m, 2^2).\]

При фиксированном $m$ величины $y_i$  независимы. Известно, что $y_1 = 0.5$, $y_2 = −1$.  Априорно известно, что место, где спряталась Маша имеет нормальное распределение, $m \sim \mN(1, 4^2)$. Нам нужно:

\begin{enumerate}
	\item Найти апостериорную плотность распределения параметра $m$.
	\item Найти апостериорные моду, медиану и математическое ожидание.
	\item Найти $\PP(m > 1 \mid y_1,y_2)$.
	\item Найти $f(y_3 \mid y_1,y_2)$ и $\E(y_3 \mid y_1,y_2)$.
\end{enumerate}
\end{problem}

\begin{sol}
Посмотрим немного подробнее на наше априорное мнение о том, где сидит Маша, $m \sim \mN(1, 4^2)$. Значение $1$ в данном случае --- наше лучшее предположение о том, где она может находиться, а $4^2$, в свою очередь, это наша степень доверия к этому предположению. Чем меньшее значение дисперсии мы берём в нашем априорном мнении, тем больше наше доверие к нему.

\indef{Делай раз! Апостериорная плотность Маши:}

\begin{multline*}
f(m \mid y_1, y_2) \propto f(y_1,y_2 \mid m) \cdot f(m) = \frac{1}{2\sqrt{2\pi}}\expp{-\frac{(0.5 - m)^2}{2 \cdot 4}} \cdot \\ \cdot \frac{1}{2\sqrt{2\pi}}\expp{-\frac{(-1 - m)^2}{2 \cdot 4}} \cdot \frac{1}{4\sqrt{2\pi}}\expp{-\frac{(m-1)^2}{2 \cdot 16}}
\end{multline*}

Воспользуемся магической силой уже привычного нам значка $\propto$ и для простоты расчётов пренебрежём кучей констант

\[ f(m \mid y_1, y_2) \propto \expp{-\frac{(0.5 - m)^2}{2 \cdot 4}} \cdot \expp{-\frac{(-1 - m)^2}{2 \cdot 4}} \cdot \expp{-\frac{(m-1)^2}{2 \cdot 16}}.\]

Сольём всё,что находится под знаком экспоненты в единое целое и упростим

\begin{multline*}
\frac{(0.5 - m)^2}{2 \cdot 4} + \frac{(-1 - m)^2}{2 \cdot 4} + \frac{(m-1)^2}{2 \cdot 16}  = \\ = \frac{ 4(m - 0.5)^2 + 4(m+1)^2 + (m-1)^2}{32} = \frac{9m^2 + 2m + 6}{32}
\end{multline*}

Используем двойную магию. С одной стороны пренебрегаем константой, с другой создаём новую для того, чтобы выделить полный квадрат. Не забываем перекинуть в знаменатель лишнюю девятку

\begin{multline*}
\expp{-\frac{9m^2 + 2m + 6}{32}} \propto \expp{-\frac{9m^2 + 2m}{32}}  = \\ = \expp{-\frac{m^2+\frac{2}{9} m}{\frac{32}{9}}}  = \expp{-\frac{m^2+ 2 \cdot \frac{1}{9} m + \frac{1}{81} - \frac{1}{81}}{\frac{32}{9}}} \propto  \\ \propto  \expp{-\frac{m^2+ 2 \frac{1}{9} m + \frac{1}{81}}{\frac{32}{9}}} = \expp{-\frac{(m + \frac{1}{9})^2}{2 \cdot (\fr{4}{3})^2}}
\end{multline*}

Видим, что параметр $m$ имеет нормальное апостериорное распределение

\[m \mid y_1, y_2 \sim \mN(-\fr{1}{9},(\fr{4}{3})^2).\]

При желании можно восстановить константу. Обратите внимания, что после того как Медведи попытались вынюхать, где находится Маша, самое вероятное её положение изменилось, а дисперсия её положения уменьшилась.

\begin{figure}[h!]
	\begin{center}
		\begin{tikzpicture}[scale = 1.2,
		every pin edge/.style={<-},
		every pin/.style={fill=yellow!50,rectangle,rounded corners=3pt,font=\small}]
		\begin{axis}[every axis plot post/.append style={
			mark=none,domain=-10:10,samples=100,smooth},
		clip=false,
		axis y line=none,
		axis x line*=bottom,
		ymin=0,
		xtick=\empty,
		]
		\addplot [blue] {\gauss{1}{2}};
		\addplot [red] {\gauss{-1/9}{4/3}};
		
		\addplot [magenta,dashed] {\gauss{0.5}{2}};
		\addplot [magenta,dashed] {\gauss{-1}{2}};
		
		\node[pin=70:{апостериорная плотность}] at (axis cs:-0.2,0.3) {};
		\node[pin=70:{априорная плотность}] at (axis cs:1.5,0.19) {};
		\node[pin=70:{вынюханная информация}] at (axis cs:1.9,0.1) {};
		\end{axis}
		\end{tikzpicture}
	\end{center}
	\caption{Информация  о Маше}
\end{figure}

Новая информация сместила априорную плотность влево и вытянула её вверх, в силу того, что Медведи вынюхали похожие вещи.

\indef{Делай два!} Мода и медиана для нормального распределения совпадают с математическим ожиданием. Мы можем использовать эти величины в качестве точечных оценок.

\indef{Делай три!} Обратите внимание, что до запуска Медведей, $\PP(m > 1) = 0.5$. После запуска, эта вероятность уменьшится,так как распределение очень сильно съедет влево.

\begin{multline*}
\PP(m > 1 \mid y_1,y_2) = 1 - \PP( m \le 1 \mid y_1,y_2) = \\ = 1 - \PP\left( \frac{m + \fr{1}{9}}{\fr{4}{3}} \le \frac{1 + \fr{1}{9}}{\fr{4}{3}} \mid  y_1, y_2 \right) = 1 - \Phi\left(\frac{10}{12}\right) \approx 0.2.
\end{multline*}

Значение функции $\Phi(z)$ можно получить, воспользовавшись таблицами для стандартной нормально распределённой случайной величины. Либо её можно найти с помощью компьютера. 


\indef{Делай четыре!} Найдём $f(y_3 \mid y_1,y_2)$ и $\E(y_3 \mid y_1,y_2)$. Будем делать это под слоганом: <<Каждой Маше по три Медведя!>>:

\begin{multline*}
f(y_3 \mid y_1,y_2)  = \int_{-\infty}^{+\infty} f(y_3, m \mid y_1,y_2)\dx{m}  = \int_{-\infty}^{+\infty}  f(y_3 \mid y_1,y_2,m) \cdot f(m \mid y_1,y_2) \dx{m}.
\end{multline*}

Под знаком интеграла мы получаем произведение модели и апостериорного распределения. Чтобы найти плотности распределение $y_3$, мы должны провести свёртку по двум нормальным распределениям

\begin{multline*}
f(y_3 \mid y_1,y_2)  = \int_{-\infty}^{+\infty} \mN(m,4)\cdot \mN \left (-\frac{1}{9},\frac{16}{9} \right) \dx{m} = \\ = \int_{-\infty}^{+\infty} 
\frac{1}{2\sqrt{2\pi \cdot 4^2}} \expp{-\frac{(y - m)^2}{2 \cdot 4^2}} \cdot 
\frac{1}{2\sqrt{2\pi \cdot (^{16}/_9)^2}} \expp{-\frac{(m + ^1/_9)^2}{2 \cdot (^{16}/_9)^2}} \dx{m} \propto \\ \propto   \int_{-\infty}^{+\infty}  \expp{-\frac{(y - m)^2}{2 \cdot 4^2} -\frac{(m + ^1/_9)^2}{2 \cdot (^{16}/_9)^2}} \dx{m}  
\end{multline*}

Если аккуратно взять этот интеграл и восстановить константу, можно получить, что $y_3 \mid y_1,y_2 \sim \mN \left(-\frac{1}{9},\frac{52}{9} \right).$ Если нам необходим точечный прогноз и в качестве функции потерь выбрано $MSE$, мы можем выбрать число $-\frac{1}{9}.$

Теперь, когда нюхательные способности третьего Машиного Медведя предсказаны, вы можете попробовать проделать всё то же самое самое, предположив, что вам вообще ничего неизвестно и $m \sim \mU(-\infty; +\infty)$. В таком случае в качестве плотности нужно будет взять $f(m) = 1$. Стоит отметить, что результат у вас, при этом, получится похожим на случай нормального априорного распределения с большой дисперсией. Эти два распределения обладают довольно большой энтропией. Из-за этого получаются схожие результаты. Также попытайтесь провернуть процедуру байесовского вывода для нормального распределения в общем случае. Формулы выйдут довольно громоздкими. Если запутаетесь, \href{https://github.com/FUlyankin/book_about_bayes}{загляните в решебник.}
\end{sol}

\end{document}