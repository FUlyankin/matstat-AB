%!TEX TS-program = xelatex
\documentclass[12pt, a4paper, oneside]{article}

% Можно вставить разную преамбулу
% пакеты для математики
\usepackage{amsmath,amsfonts,amssymb,amsthm,mathtools}  
\mathtoolsset{showonlyrefs=true}  % Показывать номера только у тех формул, на которые есть \eqref{} в тексте.

\usepackage[british,russian]{babel} % выбор языка для документа
\usepackage[utf8]{inputenc}          % utf8 кодировка

% Основные шрифты 
\usepackage{fontspec}         
\setmainfont{Linux Libertine O}  % задаёт основной шрифт документа

% Математические шрифты 
\usepackage{unicode-math}     
\setmathfont[math-style=upright]{Neo Euler} 

\setmathfont[range={\mathbb, \mathop}]{Asana-Math.otf}

%%%%%%%%%% Работа с картинками и таблицами %%%%%%%%%%
\usepackage{graphicx} % Для вставки рисунков                
\usepackage{graphics}
\graphicspath{{images/}{pictures/}}   % папки с картинками

\usepackage[figurename=Картинка]{caption}

\usepackage{wrapfig}    % обтекание рисунков и таблиц текстом

\usepackage{booktabs}   % таблицы как в годных книгах
\usepackage{tabularx}   % новые типы колонок
\usepackage{tabulary}   % и ещё новые типы колонок
\usepackage{float}      % возможность позиционировать объекты в нужном месте
\renewcommand{\arraystretch}{1.2}  % больше расстояние между строками


%%%%%%%%%% Графики и рисование %%%%%%%%%%
\usepackage{tikz, pgfplots}  % языки для графики
%\pgfplotsset{compat=1.16}

\usepackage{todonotes} % для вставки в документ заметок о том, что осталось сделать
% \todo{Здесь надо коэффициенты исправить}
% \missingfigure{Здесь будет Последний день Помпеи}
% \listoftodos --- печатает все поставленные \todo'шки


%%%%%%%%%% Внешний вид страницы %%%%%%%%%%

\usepackage[paper=a4paper, top=20mm, bottom=15mm,left=20mm,right=15mm]{geometry}
\usepackage{indentfirst}    % установка отступа в первом абзаце главы

\usepackage{setspace}
\setstretch{1.15}  % межстрочный интервал
\setlength{\parskip}{4mm}   % Расстояние между абзацами
% Разные длины в LaTeX: https://en.wikibooks.org/wiki/LaTeX/Lengths

% свешиваем пунктуацию
% теперь знаки пунктуации могут вылезать за правую границу текста, при этом текст выглядит ровнее
\usepackage{microtype}

% \flushbottom                            % Эта команда заставляет LaTeX чуть растягивать строки, чтобы получить идеально прямоугольную страницу
\righthyphenmin=2                       % Разрешение переноса двух и более символов
\widowpenalty=300                     % Небольшое наказание за вдовствующую строку (одна строка абзаца на этой странице, остальное --- на следующей)
\clubpenalty=3000                     % Приличное наказание за сиротствующую строку (омерзительно висящая одинокая строка в начале страницы)
\tolerance=10000     % Ещё какое-то наказание.

% мои цвета https://www.artlebedev.ru/colors/
\definecolor{titleblue}{rgb}{0.2,0.4,0.6} 
\definecolor{blue}{rgb}{0.2,0.4,0.6} 
%\definecolor{red}{rgb}{1,0,0.2} 
\definecolor{green}{rgb}{0, 0.6, 0}
\definecolor{purp}{rgb}{0.4,0,0.8} 

\definecolor{red}{RGB}{213,94,0}
\definecolor{yellow}{RGB}{240,228,66}


% цвета из geogebra 
\definecolor{litebrown}{rgb}{0.6,0.2,0}
\definecolor{darkbrown}{rgb}{0.75,0.75,0.75}

% Гиперссылки
\usepackage{xcolor}   % разные цвета

\usepackage{hyperref}
\hypersetup{
	unicode=true,           % позволяет использовать юникодные символы
	colorlinks=true,       	% true - цветные ссылки
	urlcolor=blue,          % цвет ссылки на url
	linkcolor=black,          % внутренние ссылки
	citecolor=green,        % на библиографию
	breaklinks              % если ссылка не умещается в одну строку, разбивать её на две части?
}

% меняю оформление секций 
\usepackage{titlesec}
\usepackage{sectsty}

% меняю цвет на синий
\sectionfont{\color{titleblue}}
\subsectionfont{\color{titleblue}}
\renewcommand{\thesection}{\arabic{section}.}


% выбрасываю нумерацию страниц и колонтитулы 
%\pagestyle{empty}

% синие круглые бульпоинты в списках itemize 
\usepackage{enumitem}

\definecolor{itemizeblue}{rgb}{0, 0.45, 0.70}

\newcommand*{\MyPoint}{\tikz \draw [baseline, fill=itemizeblue, draw=blue] circle (2.5pt);}
\renewcommand{\labelitemi}{\MyPoint}

\AddEnumerateCounter{\asbuk}{\@asbuk}{\cyrm}
\renewcommand{\theenumi}{\asbuk{enumi}}

% расстояние в списках
\setlist[itemize]{parsep=0.4em,itemsep=0em,topsep=0ex}
\setlist[enumerate]{parsep=0.4em,itemsep=0em,topsep=0ex}

% эпиграфы
\usepackage{epigraph}
\setlength\epigraphwidth{.6\textwidth}
\setlength\epigraphrule{0pt}

%%%%%%%%%% Свои команды %%%%%%%%%%

% Математические операторы первой необходимости:
\DeclareMathOperator{\sgn}{sign}
\DeclareMathOperator*{\argmin}{arg\,min}
\DeclareMathOperator*{\argmax}{arg\,max}
\DeclareMathOperator{\Cov}{Cov}
\DeclareMathOperator{\Var}{Var}
\DeclareMathOperator{\Corr}{Corr}

\DeclareMathOperator{\Pois}{Pois}
\DeclareMathOperator{\Geom}{Geom}
\DeclareMathOperator{\Exp}{Exp}

%\DeclareMathOperator{\E}{\mathbb{E}}
\DeclareMathOperator{\Med}{Med}
\DeclareMathOperator{\Mod}{Mod}
\DeclareMathOperator*{\plim}{plim}

% команды пореже
\newcommand{\const}{\mathrm{const}}  % const прямым начертанием
\newcommand{\iid}{\sim i\,i\,d\,\,}  % ну вы поняли...
\newcommand{\fr}[2]{\ensuremath{^{#1}/_{#2}}}   % особая дробь
\newcommand{\ind}[1]{\mathbbm{1}_{\{#1\}}} % Индикатор события
\newcommand{\dx}[1]{\,\mathrm{d}#1} % для интеграла: маленький отступ и прямая d

% одеваем шапки на частые штуки
\def \hb{\hat{\beta}}
\def \hs{\hat{s}}
\def \hy{\hat{y}}
\def \hY{\hat{Y}}
\def \he{\hat{\varepsilon}}
\def \hVar{\widehat{\Var}}
\def \hCorr{\widehat{\Corr}}
\def \hCov{\widehat{\Cov}}

% Греческие буквы
\def \a{\alpha}
\def \b{\beta}
\def \t{\tau}
\def \dt{\delta}
\def \e{\varepsilon}
\def \ga{\gamma}
\def \kp{\varkappa}
\def \la{\lambda}
\def \sg{\sigma}
\def \tt{\theta}
\def \Dt{\Delta}
\def \La{\Lambda}
\def \Sg{\Sigma}
\def \Tt{\Theta}
\def \Om{\Omega}
\def \om{\omega}

% Готика
\def \mA{\mathcal{A}}
\def \mB{\mathcal{B}}
\def \mC{\mathcal{C}}
\def \mE{\mathcal{E}}
\def \mF{\mathcal{F}}
\def \mH{\mathcal{H}}
\def \mL{\mathcal{L}}
\def \mN{\mathcal{N}}
\def \mU{\mathcal{U}}
\def \mV{\mathcal{V}}
\def \mW{\mathcal{W}}

% Жирные буквы
\def \mbb{\mathbb}
\def \RR{\mbb R}
\def \NN{\mbb N}
\def \ZZ{\mbb Z}
\def \PP{\mbb{P}}
\def \E{\mbb{E}}
\def \QQ{\mbb Q}

%%%%%%%%%% Теоремы %%%%%%%%%%
\theoremstyle{plain} % Это стиль по умолчанию.  Есть другие стили.
\newtheorem{theorem}{Теорема}[section]
\newtheorem{proposition}{Утверждение}[section]
\newtheorem{result}{Следствие}[theorem]
% счётчик подчиняется теоремному, нумерация идёт по главам согласованно между собой

% убирает курсив и что-то еще наверное делает ;)
\theoremstyle{definition}         
\newtheorem*{definition}{Определение}  % нумерация не идёт вообще


%%%%%%%%%% Задачки и решения %%%%%%%%%%
\usepackage{etoolbox}    % логические операторы для своих макросов
\usepackage{environ}
\newtoggle{lecture}

\newcounter{probNum}[section]  % счётчик для упражнений 
\NewEnviron{problem}[1]{%
    \refstepcounter{probNum}% увеличели номер на 1 
    {\noindent \textbf{\large \color{titleblue} Упражнение~\theprobNum~#1}  \\ \\ \BODY}
    {}%
  }

% Окружение, чтобы можно было убирать решения из pdf
\NewEnviron{sol}{%
  \iftoggle{lecture}
    {\noindent \textbf{\large Решение:} \\ \\ \BODY}
    {}%
  }
 
% выделение по тексту важных вещей
\newcommand{\indef}[1]{\textbf{ \color{green} #1}} 

\usepackage[normalem]{ulem}  % для зачекивания текста

% Если переключить в false, все solution исчезнут из pdf
\toggletrue{lecture}
%\togglefalse{lecture}



\title{\begin{center} \includegraphics[width=0.99\textwidth]{logo.png} \end{center} Шпаргалка по параметрическим критериям\footnote{Эта pdf-ка, по факту, представляет из себя немного дополненный конспект Бориса Демешева:  \url{https://github.com/bdemeshev/pr201/raw/master/probab_pset/new_el.pdf}}}
\date{ } %\today}

\begin{document} % Конец преамбулы, начало файла

\maketitle

\section{Про единственную выборку}

\subsection*{Математическое ожидание при большом числе наблюдений}

\begin{enumerate}
    \item Наблюдаем: $X_1$, $X_2$, \ldots, $X_n$;
    
    \item Предполагаем: $X_i$ независимы и одинаково распределены (не обязательно нормально), количество наблюдений $n$ велико.
    
    \item Проверяемая гипотеза: $H_0$: $\mu = \mu_0$ против $H_a$: $\mu \neq \mu_0$;
    
    \item Статистика:
    \[
    Z = \frac{\bar X - \mu_0}{se(\bar X)} = \frac{\bar X - \mu_0}{\sqrt{\frac{\hat \sigma^2}{n}}}
    \]
    
    \item При верной $H_0$ оказывается, что $Z \to \mN(0;1)$;
\end{enumerate}


\subsection*{Математическое ожидание при нормальных наблюдениях}

\begin{enumerate}
    \item Наблюдаем: $X_1$, $X_2$, \ldots, $X_n$;
    
    \item Предполагаем: $X_i$ независимы и одинаково нормально распределены $\mN(\mu; \sigma^2)$, количество наблюдений $n$ может быть мало.
    
    \item Проверяемая гипотеза: $H_0$: $\mu = \mu_0$ против $H_a$: $\mu \neq \mu_0$;
    
    \item Статистика:
    \[
    t = \frac{\bar X - \mu_0}{se(\bar X)} = \frac{\bar X - \mu_0}{\sqrt{\frac{\hat \sigma^2}{n}}}
    \]
    
    \item При верной $H_0$ оказывается, что $t \sim t_{n-1}$;
\end{enumerate}


\subsection*{Математическое ожидание при нормальных наблюдениях и известной дисперсии}

\begin{enumerate}
    \item Наблюдаем: $X_1$, $X_2$, \ldots, $X_n$, знаем величину $\sigma^2$;
    
    \item Предполагаем: $X_i$ независимы и одинаково нормально распределены $\mN(\mu; \sigma^2)$, количество наблюдений $n$ может быть мало.
    
    \item Проверяемая гипотеза: $H_0$: $\mu = \mu_0$ против $H_a$: $\mu \neq \mu_0$;
    
    \item Статистика:
    \[
    Z = \frac{\bar X - \mu_0}{\sigma_{\bar X}} = \frac{\bar X - \mu_0}{\sqrt{\frac{\sigma^2}{n}}}
    \]
    
    \item При верной $H_0$ оказывается, что $Z \sim \mN(0;1)$;
\end{enumerate}

\subsection*{Гипотеза о вероятности при наблюдениях с распределением Бернулли (0 или 1)}

\begin{enumerate}
    \item Наблюдаем: $X_1$, $X_2$, \ldots, $X_n$;
    
    \item Предполагаем: $X_i$ независимы и имеют распределение Бернулли: равны 1 с вероятностью $p$ и 0 с вероятностью $1-p$. Количество наблюдений $n$ велико.
    
    \item Проверяемая гипотеза: $H_0$: $p = p_0$ против $H_a$: $p \neq p_0$;
    
    \item Статистика:
    \[
    Z = \frac{\hat p - p_0}{se(\hat p)} = \frac{\hat p - p_0}{\sqrt{\frac{\hat p (1- \hat p)}{n}}}
    \]
    Возможен вариант этой статистики:
    
    \[
    Z = \frac{\hat p - p_0}{se(\hat p)} = \frac{\hat p - p_0}{\sqrt{\frac{p_0 (1- p_0 )}{n}}}
    \]
    
    \item При верной $H_0$ оказывается, что $Z \to \mN(0;1)$;
    
    \item Гипотеза о вероятностях является частным случаем гипотезы о математическом ожидании при большом количестве наблюдений. Можно заметить, что $\hat p = \bar X$ и $\hat \sigma^2 = \hat p (1- \hat p) \cdot \frac{n}{n-1}$. И потому также корректен вариант статистики
    
    \[
    Z = \frac{\bar X - \mu_0}{se(\bar X)} = \frac{\bar X - \mu_0}{\sqrt{\frac{\hat \sigma^2}{n}}}
    \]
\end{enumerate}
  
\subsection*{Гипотеза о дисперсии при нормальных наблюдениях}
    
\begin{enumerate}
    \item Наблюдаем: $X_1$, $X_2$, \ldots, $X_n$;
    
    \item Предполагаем: $X_i$ независимы и одинаково нормально распределены $\mN(\mu; \sigma^2)$, количество наблюдений $n$ может быть мало.
    
    \item Проверяемая гипотеза: $H_0$: $\sigma = \sigma_0$ против $H_a$: $\sigma \neq \sigma_0$;
    
    \item Статистика:
    \[
    S = \frac{\sum (X_i - \bar X)^2}{\sigma_0^2} = \frac{(n-1)\hat\sigma^2}{\sigma_0^2}
    \]
    
    \item При верной $H_0$ оказывается, что $S \sim \chi^2_{n-1}$;
\end{enumerate}

\subsection*{Гипотеза о дисперсии при нормальных наблюдениях и известном математическом ожидании}
    
\begin{enumerate}
    \item Наблюдаем: $X_1$, $X_2$, \ldots, $X_n$, знаем величину $\mu$;
    
    \item Предполагаем: $X_i$ независимы и одинаково нормально распределены $\mN(\mu; \sigma^2)$, количество наблюдений $n$ может быть мало.
    
    \item Проверяемая гипотеза: $H_0$: $\sigma = \sigma_0$ против $H_a$: $\sigma \neq \sigma_0$;
    
    \item Статистика:
    \[
    S = \frac{\sum (X_i - \bar X)^2}{\sigma_0^2} = \frac{n\hat\sigma^2}{\sigma_0^2}
    \]
    
    \item При верной $H_0$ оказывается, что $S \sim \chi^2_{n}$;
\end{enumerate}

\newpage 

\section{Про пару выборок}

\subsection*{Гипотеза о разнице ожиданий при большом количестве наблюдений}

\begin{enumerate}
    \item Наблюдаем: $X_1$, $X_2$, \ldots, $X_{n_x}$, $Y_1$, $Y_2$, \ldots, $Y_{n_y}$.
    
    Возможно, что $n_x \neq n_y$. Дисперсии $\sigma^2_x$ и $\sigma^2_y$ не знаем и не уверены, что они равны.
    
    \item Предполагаем: $X_i$ одинаково распределены между собой (не обязательно нормально),
    $Y_i$ одинаково распределены между собой, но возможно совсем не так, как $X_i$ (не обязательно нормально).
    Все величины независимы. Количества $n_x$ и $n_y$ велики.
    
    \item Проверяемая гипотеза: $H_0$: $\mu_x - \mu_y = \delta_0$ против $H_a$: $\mu_x - \mu_y \neq \delta_0$;
    
    \item Статистика:
    \[
    Z = \frac{\bar X - \bar Y - \delta_0}{se(\bar X - \bar Y)} =
    \frac{\bar X - \bar Y - \delta_0}{\sqrt{\frac{\hat \sigma^2_x}{n_x}+\frac{\hat\sigma^2_y}{n_y}}}
    \]
    
    \item При верной $H_0$ оказывается, что $Z \to \mN(0;1)$;
\end{enumerate}


\subsection*{Гипотеза о разнице ожиданий при нормальности распределения обеих выборок и известных дисперсиях}

\begin{enumerate}
    \item Наблюдаем: $X_1$, $X_2$, \ldots, $X_{n_x}$, $Y_1$, $Y_2$, \ldots, $Y_{n_y}$.
    Возможно, что $n_x \neq n_y$. Дисперсии $\sigma^2_x$ и $\sigma^2_y$ знаем. Возможно, что дисперсии не равны.
    
    \item Предполагаем: $X_i$ одинаково распределены между собой $\mN(\mu_x, \sigma^2_x)$,
    $Y_i$ одинаково распределены между собой $\mN(\mu_y, \sigma^2_y)$.
    Все величины независимы. Количества $n_x$ и $n_y$ любые.
    
    \item Проверяемая гипотеза: $H_0$: $\mu_x - \mu_y = \delta_0$ против $H_a$: $\mu_x - \mu_y \neq \delta_0$;
    
    \item Статистика:
    \[
    Z = \frac{\bar X - \bar Y - \delta_0}{\sigma_{\bar X - \bar Y}} =
    \frac{\bar X - \bar Y - \delta_0}{\sqrt{\frac{\sigma^2_x}{n_x}+\frac{\sigma^2_y}{n_y}}}
    \]

    \item При верной $H_0$ оказывается, что $Z \sim \mN(0;1)$;
\end{enumerate}


\subsection*{Гипотеза о разнице ожиданий при нормальности распределения обеих выборок и неизвестных но равных дисперсиях}

\begin{enumerate}
    \item Наблюдаем: $X_1$, $X_2$, \ldots, $X_{n_x}$, $Y_1$, $Y_2$, \ldots, $Y_{n_y}$.
    Возможно, что $n_x \neq n_y$. Дисперсии $\sigma^2_x$ и $\sigma^2_y$ равны, но неизвестны.
    
    \item Предполагаем: $X_i$ одинаково распределены между собой $\mN(\mu_x, \sigma^2)$,
    $Y_i$ одинаково распределены между собой $\mN(\mu_y, \sigma^2)$.
    Все величины независимы. Количества $n_x$ и $n_y$ любые.
    
    \item Проверяемая гипотеза: $H_0$: $\mu_x - \mu_y = \delta_0$ против $H_a$: $\mu_x - \mu_y \neq \delta_0$;
    
    \item Статистика:
    \[
    t = \frac{\bar X - \bar Y - \delta_0}{se(\bar X - \bar Y)} =
    \frac{\bar X - \bar Y - \delta_0}{\sqrt{\frac{\hat \sigma^2}{n_x}+\frac{\hat\sigma^2}{n_y}}},
    \]
    где
    \[
    \hat \sigma^2 = \frac{\sum (X_i - \bar X)^2 + \sum (Y_i - \bar Y)^2 }{n_x + n_y - 2}
    \]
    
    \item При верной $H_0$ оказывается, что $t \sim t_{n_x+n_y-2}$;
\end{enumerate}


\subsection*{Гипотеза о разнице ожиданий в связанных парах}

\begin{enumerate}
    \item Наблюдаем: $X_1$, $X_2$, \ldots, $X_{n}$, $Y_1$, $Y_2$, \ldots, $Y_{n}$.
    Количество $X_i$ и $Y_i$ одинаковое.
    \item Предполагаем: внутри пары $X_i$ и $Y_i$ зависимы, а наблюдения с разными номерами независимы.
    Рассматриваем разницу $D_i = X_i - Y_i$ и получаем одномерную выборку.
    Величины $D_i$ независимы и одинаково распределены.
    Возможно три описанных ранее случая :)
    Здесь для примера рассмотрим случай, когда $D_i \sim \mN(\mu_d, \sigma^2_d)$ с неизвестной дисперсией.
    
    \item Проверяемая гипотеза: $H_0$: $\mu_d = \mu_0$ против $H_a$: $\mu_d \neq \mu_0$;
    
    \item Статистика:
    \[
    t = \frac{\bar D - \mu_d}{se(\bar D)} =
    \frac{\bar X - \bar Y - \mu_d}{\sqrt{\frac{\hat \sigma^2_d}{n}}},
    \]
    где
    \[
    \hat \sigma^2_d = \frac{\sum (D_i - \bar D)^2 }{n - 1} = \frac{\sum (X_i - Y_i - (\bar X - \bar Y))^2 }{n - 1}
    \]
    
    \item При верной $H_0$ оказывается, что $t \sim t_{n-1}$;
\end{enumerate}

\subsection*{Гипотеза о равенстве дисперсий при нормальности распределения обеих выборок}

\begin{enumerate}
    \item Наблюдаем: $X_1$, $X_2$, \ldots, $X_{n_x}$, $Y_1$, $Y_2$, \ldots, $Y_{n_y}$.
    Возможно, что $n_x \neq n_y$. Дисперсии $\sigma^2_x$ и $\sigma^2_y$ не знаем. Возможно, что дисперсии не равны.
    
    \item Предполагаем: $X_i$ одинаково распределены между собой $\mN(\mu_x, \sigma^2)$,
    $Y_i$ одинаково распределены между собой $\mN(\mu_y, \sigma^2)$.
    Все величины независимы. Количества $n_x$ и $n_y$ любые.
    
    \item Проверяемая гипотеза: $H_0$: $\sigma_x = \sigma_y$ против $H_a$: $\sigma_x \neq \sigma_y$;
    
    \item Статистика:
    \[
    F = \frac{\hat \sigma^2_x}{\hat \sigma^2_y}
    \]
    
    \item При верной $H_0$ оказывается, что $F \sim F_{n_x-1, n_y - 1}$;
\end{enumerate}


\end{document}

